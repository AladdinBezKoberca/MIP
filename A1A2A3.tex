% Metódy inžinierskej práce

\documentclass[10pt,slovak,a4paper]{coursepaper}

\usepackage[slovak]{babel}
%\usepackage[T1]{fontenc}
\usepackage[IL2]{fontenc} % lepšia sadzba písmena Ľ než v T1
\usepackage[utf8]{inputenc}
\usepackage{graphicx}
\usepackage{url} % príkaz \url na formátovanie URL
\usepackage{hyperref} % odkazy v texte budú aktívne (pri niektorých triedach dokumentov spôsobuje posun textu)

\usepackage{cite}
%\usepackage{times}

\pagestyle{headings}

\title{Názov\thanks{Semestrálny projekt v predmete Metódy inžinierskej práce, ak. rok 2015/16, vedenie: Meno Priezvisko}} % meno a priezvisko vyučujúceho na cvičeniach

\author{Adam Džubák\\[2pt]
	{\small Slovenská technická univerzita v Bratislave}\\
	{\small Fakulta informatiky a informačných technológií}\\
	{\small \texttt{xdzubak@stuba.sk}}
	}

\date{\small 28. októbra 2015} % upravte



\begin{document}

\maketitle

\begin{abstract}
\ldots
\end{abstract}



\section{AA-A1}
01 nic  \\
02  9.10.2025  \\
03 Predikcia a signalizácia potreby ošetrenia viniča \\
04 VITIPRED \\
05 Agroinformatika a automatiácia v poľnohospodárstve \\
06 Vývoj \\
07 22.6.2025 \\
08 22.11.2025 \\
09 Projekt sa zameriava na vývoj a implementáciu prediktívnych modelov, ktoré umožnia
včasnú signalizáciu potreby ošetrenia viniča proti chorobám a škodcom. Cieľom je prepojiť aktuálne meteorologické údaje (teploty a zrážky) s dlhodobými klimatickými trendmi (tridsaťročné týždenné priemery zrážok a teplôt) a prediktívnym modelom počasia (trojtýždenný kĺzavý priemer zrážok a teplôt), čím sa vytvorí nástroj pre presné a efektívne
rozhodovanie o potrebe chemického ošetrenia viniča. Výstupom projektu bude systém,
ktorý dokáže na základe analýzy teploty a zrážok predpovedať pravdepodobnosť výskytu
najvýznamnejších chorôb viniča (botrytída, peronospóra, guignardia, obaľovače). Projekt
využíva metódy štatistickej analýzy, strojového učenia a modelovania klimatických dát,
pričom čerpá z vedeckých poznatkov v oblasti fyziológie viniča a epidemiológie chorôb
rastlín a kľúčových fenologických štádií viniča. Model bude testovaný na dátach malokarpatskej vinohradníckej oblasti s cieľom zabezpečiť jeho prenositeľnosť do rôznych klimatických podmienok severnej pologule. Očakávaným prínosom projektu je vytvorenie nástroja
pre včasné a rýchle informovanie vinohradníkov o potrebe chemického ošetrenia viniča
minimalizujúceho riziko jeho ochorenia. Predikčný systém tak prispeje k udržateľnému
pestovaniu viniča a zvýšeniu kvality produkcie vína. Projekt spája odborné poznatky z
meteorológie, biológie a dátovej analýzy a spracovania. Zároveň vytvára základ pre moderný prístup k vinohradníckemu manažmentu v podmienkach klimatickej zmeny. \\
10  Slovenská Technická Univerzita  \\
11 109 800 eur \\
12 nic \\
13 119 800 eur \\

\section{AA-A2}
01 Fakulta Infomatiky a Informačných Technológií \\
02 FIIT \\
03 Ilkovičova 2 \\
04 00397687 \\
05 Ministerstvo školstva, vedy, výskumu a športu SR\\
06 Štátna príspevková organizácia \\
07  \\
08 prof. Ing. Maximilián Strémy, PhD.\\
09 prof. Ing. Ivan Kotuliak, PhD.\\

\section{AA-A3}
Adam Džubák   	10.5.2006	1440\\
Michal Fabrici   	16.12.2005	1440\\
Lukáš Čimo 	 13.1.2006		1440\\
Ondrej Fársky  	 25.9.2005	1440\\




\input{example.tex}
% týmto sa generuje zoznam literatúry z obsahu súboru literatura.bib podľa toho, na čo sa v článku odkazujete
\bibliography{literatura}
\bibliographystyle{abbrv} % prípadne alpha, abbrv alebo hociktorý iný
\end{document}
